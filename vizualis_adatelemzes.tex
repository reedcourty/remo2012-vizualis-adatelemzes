\documentclass[a4paper,10pt,titlepage]{article}
% PREUMBULUM
\usepackage[utf8]{inputenc}
\usepackage[T1]{fontenc}

\usepackage{a4wide} 
\usepackage{times}

\usepackage[magyar,english]{babel}

% Forráskódoknak:
\usepackage{listings}

% Tartalomjegyzék:
\usepackage{tocbibind}

\usepackage[usenames,dvipsnames]{color}

% Hogy legyen képünk:
\usepackage{graphicx}

\usepackage{hyperref}
\hypersetup{
    bookmarks=true,
    unicode=true,
    colorlinks=true,
    linkcolor=RoyalBlue,
    citecolor=RoyalBlue,
    filecolor=RoyalBlue,
    urlcolor=RoyalBlue,
}

% Táblázatoknak:
\usepackage{colortbl}

% Ha kell matek:
\usepackage{amssymb,amsmath}

\usepackage{subfig}
\usepackage{verbatim} % Hogy lehessen blokkkommentezni

\usepackage{ftsrgtemplate}

%###########################################
% Saját eszközök:
%
\definecolor{todobgszin}{rgb}{0.64,0.78,0.22}
\definecolor{todofrszin}{rgb}{0.00,0.50,0.00}

\newcommand{\todo}[1]{\fcolorbox{todofrszin}{todobgszin}{\emph{TODO: #1}}}
\newcommand{\angolul}[1]{\foreignlanguage{english}{#1}}

\newenvironment{sajat_itemize}
{
	\begin{itemize}
	\setlength{\itemsep}{0pt}
}
{
	\end{itemize}
}

\begin{document}
% Dokumentumtörzs

\selectlanguage{magyar}

% Címoldal:
\begin{titlepage}

\title{\colorbox{ftsrg_color}{\parbox{\textwidth}{%
  \vskip40pt
  \leftskip10pt\rightskip10pt
  \center{\color{white}{Rendszermodellezés - 2. gyakorlat \\ Vizuális adatelemzés}}
  \vskip40pt
 }
}}
\author{Nádudvari György \\ < ulqp9p@gmail >}
\date{\today}

\end{titlepage}
\maketitle

\section{Bevezetés}

A gyakorlat célja, hogy nagy vonalakban bemutassuk a vizuális adatelemzést és az ahhoz kapcsolódó előkészületeket, használható alkalmazásokat.

\todo{Bővíteni!}

%------------------------------------------------------------------------------
\section{A TPC benchmarkok}
\subsection{A TPC benchmark}
A TPC benchmarkokat a nonprofit \textit{Transaction Processing Performance Council} dolgozta ki, amelyek főleg web és adatbázis rendszerek teljesítményét hivatottak mérni. További információk: \href{http://www.tpc.org}{http://www.tpc.org}
\subsection{A TPC-C benchmark}
A TPC-C benchmark egy összetett OLTP (On-line Transaction Processing) benchmark, amely különböző típusú és komplexitású konkurens tranzakciók percenkénti számát méri (tpmC).

\end{document}