\documentclass[a4paper,10pt,titlepage]{article}
% PREUMBULUM
\usepackage[utf8]{inputenc}
\usepackage[T1]{fontenc}

\usepackage{a4wide} 

% Használjunk inkább Arial betűtípust:
%\usepackage{times}
\renewcommand{\rmdefault}{phv} % Arial
\renewcommand{\sfdefault}{phv} % Arial

\usepackage[magyar,english]{babel}

% Forráskódoknak:
\usepackage{listings}

% Tartalomjegyzék:
\usepackage{tocbibind}

\usepackage[usenames,dvipsnames]{color}

% Hogy legyen képünk:
\usepackage{graphicx}

%###############################################################################
\newcommand{\szerzo}{Nádudvari György}
\newcommand{\szerzomail}{ulqp9p@gmail}
\newcommand{\cim}{Rendszermodellezés - 2. gyakorlat \\ Vizuális adatelemzés}
\newcommand{\targy}{rendszermodellezés, vizuális adatelemzés}
\newcommand{\kulcsszavak}{rendszermodellezés, vizuális adatelemzés, adattisztítás, Mondrian, TPC, TPC-C}

\usepackage{hyperref}
\hypersetup{
    unicode=true,
    colorlinks=true,
    linkcolor=RoyalBlue,
    citecolor=RoyalBlue,
    filecolor=RoyalBlue,
    urlcolor=RoyalBlue,
    pdftitle={\cim},        % title
    pdfauthor={\szerzo},    % author
    pdfsubject={\targy}, % subject of the document
    pdfcreator={},   % creator of the document
    pdfproducer={LaTeX, TexMaker},    % producer of the document
    pdfkeywords={\kulcsszavak},    % list of keywords
}

\usepackage{url}

% Táblázatoknak:
\usepackage{colortbl}

% Ha kell matek:
\usepackage{amssymb,amsmath}

\usepackage{verbatim} % Hogy lehessen blokkkommentezni

% Egymás melleti képekhez:
\usepackage{subfig}

\usepackage{ftsrgtemplate}

\setlength{\parindent}{12pt} % magyar nyelvű dokumentumokban jellemző
\setlength{\parskip}{0pt}    % magyar nyelvű dokumentumokban jellemző

%###########################################
% Saját eszközök:
%
\definecolor{todobgszin}{rgb}{0.64,0.78,0.22}
\definecolor{todofrszin}{rgb}{0.00,0.50,0.00}
\definecolor{alice_blue}{rgb}{0.94,0.97,1.00}
\definecolor{bone}{rgb}{0.89,0.85,0.79}
\definecolor{eggshell}{rgb}{0.94,0.92,0.84}
\definecolor{light_yellow}{rgb}{1.00,1.00,0.88}
\definecolor{buff}{rgb}{0.94,0.86,0.51}

%\newcommand{\todo}[1]{\fcolorbox{todofrszin}{todobgszin}{\emph{TODO: #1}}}
\newcommand{\angolul}[1]{\foreignlanguage{english}{#1}}

\newcommand{\todo}[1]{
    \vfill
    \begingroup % Csinálunk egy csoportot, hogy az identálás csak erre vonatkozzon
        \setlength{\parindent}{0cm} % Beállítjuk, hogy teljes szélességű legyen a dobozunk a bekezdéstől függetlenül
        \fcolorbox{todofrszin}{todobgszin}{
            \parbox{\textwidth}{
                \vskip10pt
                \leftskip10pt
                \rightskip10pt
            
                \emph{TODO: #1}
  
                \vskip10pt
            }
        }
    \endgroup
    \vfill
}

\newcommand{\appcomment}[1]{
    \vfill
    \begin{center}
    \begingroup % Csinálunk egy csoportot, hogy az identálás csak erre vonatkozzon
        \setlength{\parindent}{0cm} % Beállítjuk, hogy teljes szélességű legyen a dobozunk a bekezdéstől függetlenül
        \setlength{\textwidth}{12cm}
        \fcolorbox{bone}{eggshell}{
            \parbox{\textwidth}{
                \vskip5pt
                \leftskip5pt
                \rightskip5pt
                
                \small
                
                \textbf{Megjegyzés}
                \newline           
                                
                #1
  
                \vskip5pt
            }
        }
    \endgroup
    \end{center}
    \vfill
}


%###############################################################################
\newcommand{\oszlopnev}[1]{\fcolorbox{bone}{buff}{\texttt{#1}}}
\newcommand{\menuelem}[1]{\fcolorbox{bone}{alice_blue}{\emph{#1}}}
\newcommand{\billkomb}[1]{\textit{#1}}

\newenvironment{sajat_itemize}
{
	\begin{itemize}
	\setlength{\itemsep}{0pt}
}
{
	\end{itemize}
}

\begin{document}
% Dokumentumtörzs

\selectlanguage{magyar}

% Címoldal:
\begin{titlepage}

\title{\colorbox{ftsrg_color}{\parbox{\textwidth}{%
  \vskip40pt
  \leftskip10pt\rightskip10pt
  \center{\textcolor{white}{Rendszermodellezés - 2. gyakorlat \\ Vizuális adatelemzés}}
  \vskip40pt
 }
}}
\author{\szerzo \\ < \szerzomail >}
\date{\today}

\end{titlepage}
\maketitle

% Nem akarom, hogy megjelenjen a tartalomjegyzékben a Tartalomjegyzék:
\section*{Tartalomjegyzék}
\makeatletter
\@starttoc{toc}
\makeatother

\newpage

\section{Bevezetés}

A gyakorlat célja, hogy nagy vonalakban bemutassuk a vizuális adatelemzést és az ahhoz kapcsolódó előkészületeket, használható alkalmazásokat.

\todo{Bővíteni!}

%------------------------------------------------------------------------------
\section{Benchmarking}
\subsection{Mik azok a benchmarkok?}

A benchmarkok olyan részletesen specifikált, dokumentált, megismételhető mérések, amelyek segítségével hardver- és/vagy szoftverkonfigurációk teljesítményeinek összehasonlítását hivatottak segíteni. A benchmark környezet specifikációjában rögzített a

\begin{sajat_itemize}
\item hardver komponensek listája,
\item szoftver komponensek listája,
\item üzemviszonyok,
\item ütemezés,
\item dokumentáció.
\end{sajat_itemize} 

\subsection{A TPC benchmarkok}
A TPC benchmarkokat a nonprofit \textit{Transaction Processing Performance Council} dolgozta ki, amelyek főleg web és adatbázis rendszerek teljesítményét hivatottak mérni.

További információkat a szervezet honlapján (\url{http://www.tpc.org}) találhatunk.

\subsection{A TPC-C benchmark}
A TPC-C benchmark egy összetett OLTP (On-line Transaction Processing) benchmark, amely különböző típusú és komplexitású konkurens tranzakciók percenkénti számát méri (tpmC). A tranzakciók lefutási idejére rögzített felső korlát van.

A mérési környezet része egy mintaadatbázis, amely ügyfeleket és megrendeléseket tartalmaz. A mérés kialakításánál igyekeztek ACID körülményeket, és a felhasználók gondolkodási idejét is szimulálni. A mért adatok az áteresztőképesség (tpmC) és a hatékonyság (ár/tpmC). 

A segédlet további részében ezeket az eredményeket fogjuk felhasználni.
 
%------------------------------------------------------------------------------
\section{Adattisztítás}

Általában minden adatelemzést megelőz az adatok tisztítása. Gyakran olyan adathalmazt kell feldolgoznunk, amely számunkra felesleges, vagy rosszul formázott adatokat tartalmaznak. Ezeket nekünk kell megfelelő módszerekkel előkészíteni oly módon, hogy abból az elemzés során új információkat, összefüggéseket tudjuk megmutatni.

A rosszul formázott adatokra lehet példa a nem megfelelő dátum, vagy a túl sok különböző értéket tartalmazó rekordok, amelyeket érdemes aggregálni a könnyebb megjeleníthetőség, feldolgozhatóság miatt. Az esetlegesen hiányos mezőknek olyan értékeket kell biztosítani, amelyek nem módosítják az elemzés eredményét, ugyanakkor biztosítják azt, hogy a feldolgozó rendszer ne fusson hibába azok hiányából eredően.

Az adattisztítás tipikusan olyan feladat, amelyet csak emberi gondolkodással lehet elvégezni, ám a későbbiekben az esetlegesen ismétlődő részeket érdemes lehet automatizálni.

\subsection{Manuális adattisztítás}

Kisméretű adathalmaz esetén akár kézzel is megtisztíthatjuk azt. Ehhez csak ismernünk kell az adatok jelentését, hogy miből származnak, milyen formátumban állnak rendelkezésre az egyes mezők. Egyszerűsödik a munka, ha olyan elrendezésben érhetőek el, hogy azt egy táblázatkezelő alkalmazásba tudjuk importálni. 

Mostani példánkban a TPC-C eredményeinek átalakításán megyünk végig, áttekintve ezzel az adattisztítás fontosabb lépéseit. Ezek az eredmények letölthetőek a TPC honlapjáról: \url{http://www.tpc.org/downloaded\_result\_files/tpcc\_results.txt}

\Aref{fig:tpc_raw}.~ábrán látható az adathalmazunk, miután importáltuk kedvenc táblázatkezelőnkbe.

\begin{figure}[h!]
\centering
\includegraphics[width=1.00\textwidth]{figures/tpc_raw.png}
\caption{A tisztítatlan adathalmaz importálás után \label{fig:tpc_raw}}
\end{figure}

A tisztítás fontosabb lépései:

\begin{sajat_itemize}
\item importálás táblázatkezelőbe
\item felesleges sorok eltávolítása
\item felesleges oszlopok eltávolítása (itt pl. \emph{spec. revision}, \emph{withdraw} stb.)
\item tizedespont vs. vessző átalakítások a számokat tartalmazó cellákban
\item új származtatott oszlopok felvétele (itt pl. \emph{Price/Perf (USD)} és \emph{Total Sys. Cost (USD)})
\item adatok aggregálása (itt pl. \emph{Database Software}, \emph{Operating System})
\item exportálás (esetünkben tabulált értékek, hogy a később bemutatásra kerülő Mondriánba tudjuk importálni)
\end{sajat_itemize}

\Aref{fig:tpc_cleaned_001}., \ref{fig:tpc_cleaned_002}. és \ref{fig:tpc_cleaned_003}.~ábrán látható a tisztított adathalmazunk.

\begin{figure}[h!]
\centering
\includegraphics[width=0.70\textwidth]{figures/tpc_cleaned_001.png}
\caption{A tisztított adathalmazunk \label{fig:tpc_cleaned_001}}
\end{figure}

\begin{figure}[h!]
\centering
\includegraphics[width=0.60\textwidth]{figures/tpc_cleaned_002.png}
\caption{A tisztított adathalmazunk \label{fig:tpc_cleaned_002}}
\end{figure}

\begin{figure}[h!]
\centering
\includegraphics[width=1.00\textwidth]{figures/tpc_cleaned_003.png}
\caption{A tisztított adathalmazunk \label{fig:tpc_cleaned_003}}
\end{figure}

\subsection{Automatikus adattisztítás}

Nagyobb méretű adathalmaz esetén nem alkalmazható a kézi tisztítás, ehelyett ilyenkor céleszközöket használunk. Mivel napjainkban egyre elterjedtebb az adatfeldolgozás, adatbányászat, ezért igen széles választék áll rendelkezésünkre a feladat elvégzésére.

Az egyik lehetőség a szkript vagy programozási nyelvek használata, mint pl. a Python\footnote{\url{http://python.org}} vagy az R nyelv\footnote{\url{http://www.r-project.org/}}. Az utóbbi ugyan egy statisztikai nyelv, de gyakran használják adatelemzésre is (az egyik legelterjedtebb fejlesztőkörnyezete \aref{fig:RStudio}.~ábrán látható). Egy másik típusba tartozó céleszközök a grafikusan tervezett adatfeldolgozási folyamatok, mint pl. a KNIME\footnote{\url{http://www.knime.org/}} (\ref{fig:knime}.~ábra). 

\begin{figure}[h!]
\centering
\includegraphics[width=1.00\textwidth]{figures/RStudio.png}
\caption{RStudio, az R nyelv egyik fejlesztőkörnyezete \label{fig:RStudio}}
\end{figure}

\begin{figure}[h!]
\centering
\includegraphics[width=1.00\textwidth]{figures/knime.png}
\caption{A KNIME adatfeldolgozó rendszer \label{fig:knime}}
\end{figure}

\section{Mondrian}

\todo{Leírás róla, hogy mi is ez}

%------------------------------------------------------------------------------
\section{Feladatok}

\subsection{Adatok betöltés a Mondrian alkalmazásba}

Indítsuk el a Mondriant, majd \menuelem{File $\rightarrow$ Open}, és keressük meg a TPC-C eredményeit tartalmazó állományt (\textsl{tpcc.tsv}). Sikeres megnyitás utána \aref{fig:mondrian_tpcc_betoltve}.~ábrán látható mezőknek kell szerepelni az alkalmazás fő ablakában.
Ezek után már elkezdhetjük az elemzést.

\appcomment{
Egyes adatforrások esetében előfordulhat, hogy a Mondrian tévesen határozza meg azok típusát (diszkrét/folytonos értékek). Ilyenkor nekünk kell azt helyesen beállítani az oszlopra történő dupla kattintással.
}

\begin{figure}[h!]
\centering
\includegraphics[width=0.30\textwidth]{figures/mondrian_tpcc_betoltve.png}
\caption{A Mondrianba betöltött adathalmaz oszlopai \label{fig:mondrian_tpcc_betoltve}}
\end{figure}

\subsection{Mely években megjelent konfigurációkat tartalmazza a benchmark? Mennyire használható/releváns napjainkban?}
\subsubsection*{Megoldás}
Válasszuk ki az \oszlopnev{Availability\_Date\_year} oszlopot (egy kattintás az oszlopra), majd \menuelem{Plot $\rightarrow$ Barchart}. \Aref{fig:mondrian_Availability_Date_year_barchart}.~ábrán látható, hogy 2009-től megfogyatkozott a feltöltött eredmények száma.

\begin{figure}[h!]
\centering
\includegraphics[width=0.30\textwidth]{figures/mondrian_Availability_Date_year_barchart.png}
\caption{Az eredményhalmazban szereplő konfigurációk száma éves bontásban \label{fig:mondrian_Availability_Date_year_barchart}}
\end{figure}
  
\appcomment{A Mondrian alapbeállítása esetén a kijelölés színe a piros, ám ez a későbbiek során esetleg zavaró lehet ezért érdemes átállítani egy másik tetszőleges, jól látható színre. Ezt az \menuelem{Options $\rightarrow$ Preferences ... $\curvearrowright$ highlighting color} változtatásával tudjuk elkövetni.}

\subsection{Az egyes beszállítók mely években voltak aktívak? Mely beszállító a nagy játékos?}
\subsubsection*{Megoldás}
Válasszuk ki a \oszlopnev{Company} oszlopot, majd \menuelem{Plot $\rightarrow$ Barchart}. Az egyes beszállítók neveire kattintva az éveket tartalmazó ablakban láthatjuk az aktív éveket.
\subsubsection*{Megállapítások, megjegyzések}
A HP minden évben adott ki konfigurációt (kivétel 2012, de még nincs vége az évnek), ahogy az \aref{fig:mondrian_Availability_Date_year_barchart_Company_barchart_HP}.~ábrán is látható, míg az Oracle csak az utóbbi két évben\footnote{Persze itt ez a megállapítás sántít. Előfordulhat, hogy minden beszállító minden évben piacra dobott új termékeket, ám azok eredményeit nem küldte be a TPC adatbázisába.}.

\begin{figure}[h!]
  \centering
  \subfloat[Az egyes beszállítók eloszlása]{\label{fig:mondrian_Company_barchart}\includegraphics[width=0.30\textwidth]{figures/mondrian_Company_barchart.png}}
  ~~~~~ % Térköz az ábrák között.
  \subfloat[A HP aktív évei]{\label{fig:mondrian_Availability_Date_year_barchart_Company_barchart_HP}\includegraphics[width=0.30\textwidth]{figures/mondrian_Availability_Date_year_barchart_Company_barchart_HP.png}}
  \caption{Az egyes beszállítók aktivitása}
  \label{fig:beszallitok_aktivitasa}
\end{figure}

\subsection{Ha cégünk igényeit egy alacsonyabb teljesítményű konfigurációval is ki tudjuk elégíteni, akkor mely beszállítók közül válasszunk?}
\subsubsection*{Megoldás}
Nézzük meg a \emph{tpmC} eloszlását. Ehhez válasszuk ki a \oszlopnev{tpmC} oszlopot, majd \menuelem{Plot $\rightarrow$ Histogram}. A bal szélső oszlopra kattintva a \emph{Company} nézetén láthatjuk a szóba jöhető beszállítókat (\aref{fig:mondrian_tpmc_barchart_Company_barchart_low}.~ábra színezett része). A hisztogram bin szélességét a \billkomb{Ctrl + UP/DOWN} billentyűkkel módosíthatjuk.
\subsubsection*{Megállapítások, megjegyzések}
\Aref{fig:mondrian_tpmc_barchart}.~ábra jobb szélső oszlopa az Oracle csúcskategóriás klasztere.

\begin{figure}[h!]
  \centering
  \subfloat[Az teljesítmények eloszlás módosított bin szélesség után]{\label{fig:mondrian_tpmc_barchart}\includegraphics[width=0.30\textwidth]{figures/mondrian_tpmc_barchart.png}}
  ~~~~~ % Térköz az ábrák között.
  \subfloat[A beszállítók]{\label{fig:mondrian_tpmc_barchart_Company_barchart_low}\includegraphics[width=0.30\textwidth]{figures/mondrian_tpmc_barchart_Company_barchart_low.png}}
  \caption{Az egyes beszállítók teljesítménye}
  \label{fig:beszallitok_teljesitmenye}
\end{figure}

\subsection{Mit lehet megállapítani a teljesítmény változásáról?}
\subsubsection*{Megoldás}
Válasszuk ki az \oszlopnev{Availability\_Date\_year} vagy \oszlopnev{Availability\_Date\_day}, majd a \oszlopnev{tpmC} oszlopokat (a \billkomb{Ctrl} gomb nyomva tartásával tudunk egyszerre több elemet kijelölni), \menuelem{Plot $\rightarrow$ Scatterplot}.
\subsubsection*{Megállapítások, megjegyzések}
Látható a teljesítmény növekedése, de néhány csúcsteljesítményű szerver kivételével inkább az alsó sávban helyezkednek el.

Illesszünk regressziós görbét a pontokra. Az \oszlopnev{Availability\_Date\_year - tpmC} scatterplotra jobb gombbal kattintva \menuelem{smoothers $\rightarrow$ ls-line}. \Aref{fig:mondrian_Availability_Date_day_tpmC_scatterplot_ls_line}.~ábra jobb alsó sarkában látható egy lineáris egyenes, amely az LS-line\footnote{\url{http://en.wikipedia.org/wiki/Least\_squares}} regressziós függvény eredménye, míg \aref{fig:mondrian_Availability_Date_day_tpmC_scatterplot_splines}.~ábrán egy másik algoritmus kimenete látható.

\begin{figure}[h!]
  \centering
  \subfloat[Idő - tpmc pontdiagram]{\label{fig:mondrian_Availability_Date_day_tpmC_scatterplot}\includegraphics[width=0.30\textwidth]{figures/mondrian_Availability_Date_day_tpmC_scatterplot.png}}
  ~~~ % Térköz az ábrák között.
  \subfloat[LS-line regressziós görbe alkalmazása]{\label{fig:mondrian_Availability_Date_day_tpmC_scatterplot_ls_line}\includegraphics[width=0.30\textwidth]{figures/mondrian_Availability_Date_day_tpmC_scatterplot_ls_line.png}}
  ~~~ % Térköz az ábrák között.
  \subfloat[Splines regressziós görbe alkalmazása]{\label{fig:mondrian_Availability_Date_day_tpmC_scatterplot_splines}\includegraphics[width=0.30\textwidth]{figures/mondrian_Availability_Date_day_tpmC_scatterplot_splines.png}}
  \caption{A teljesítmények változása az idő haladtával}
  \label{fig:beszallitok_teljesitmenye}
\end{figure}

\appcomment{A Mondrián -- mint ahogy az már korábban említésre került --, képes az R környezethez kapcsolódni, amennyiben az \emph{RServe} (A projekt honlapja: \url{http://rforge.net/Rserve/}) csomag telepítve van. Ennek hiánya esetén csak az LS-line funkció érhető el a regressziós görbe típusok közül, mivel a többi megvalósításához bonyolultabb számítások szükségesek.}

\appcomment{Ha valamelyik pont fölé visszük a kurzort és közben nyomva tartjuk a \billkomb{Ctrl}-t akkor kiírja a ponthoz tartozó értékeket.}

\subsubsection*{Ugyanez Boxplot alkalmazásával}
Válasszuk ki az \oszlopnev{Availability\_Date\_year}, majd a \oszlopnev{tpmC} oszlopokat, \menuelem{Plot $\rightarrow$ Boxplot y by x}. \Aref{fig:mondrian_Availability_Date_year_tpmC_boxplot}.~ábrán megfigyelhető a mediánok értékeinek növekedése.

\begin{figure}[h!]
\centering
\includegraphics[width=0.80\textwidth]{figures/mondrian_Availability_Date_year_tpmC_boxplot.png}
\caption{Az \emph{Availability\_Date\_year} - \emph{tpmC} kapcsolata dobozdiagramon ábrázolva} \label{fig:mondrian_Availability_Date_year_tpmC_boxplot}
\end{figure}

\subsection{Hogyan alakulnak a tranzakciós és összköltségek?}
\subsubsection*{Megoldás}
Válasszuk ki az \oszlopnev{Availability\_Date\_year}, majd a \oszlopnev{Price\_per\_Perf\_in\_USD} oszlopokat, \menuelem{Plot $\rightarrow$ Scatterplot}. Ezután az \oszlopnev{Availability\_Date\_year}, \oszlopnev{Total\_Sys\_Cost\_in\_USD} párost hasonló módon. Tegyünk regressziós görbét az ábrákra (jobb klikk az \emph{Availability\_Date\_year - Price\_per\_Perf\_in\_USD} grafikonra, és például \menuelem{smoothers $\rightarrow$ splines})!

\begin{figure}[h!]
  \centering
  \subfloat[Beszállítók, kiemelve az IBM]{\label{fig:mondrian_Company_ADY_P_and_T_IBM}\includegraphics[width=0.30\textwidth]{figures/mondrian_Company_ADY_P_and_T_IBM.png}}
  ~~~ % Térköz az ábrák között.
  \subfloat[Évek - teljesítmény alapú költség, kiemelve az IBM]{\label{fig:mondrian_Availability_Date_year_Price_per_Perf_in_USD_splines}\includegraphics[width=0.30\textwidth]{figures/mondrian_Availability_Date_year_Price_per_Perf_in_USD_splines.png}}
  ~~~ % Térköz az ábrák között.
  \subfloat[Évek - teljes költség, kiemelve az IBM]{\label{fig:mondrian_Availability_Date_year_Total_Sys_Cost_in_USD_splines}\includegraphics[width=0.30\textwidth]{figures/mondrian_Availability_Date_year_Total_Sys_Cost_in_USD_splines.png}}
  \caption{A költségek alakulása az idő haladtával}
  \label{fig:koltsegek_alakulasa}
\end{figure}

\Aref{fig:koltsegek_alakulasa}.~ábrán láthatjuk, hogy a tranzakciós költségek csökkennek, viszont a teljes költségek inkább növekednek.

\appcomment{Tetszőleges kijelölés után a Mondrian újraszámolja a regressziós görbéket a kijelölt pontok alapján, és a kijelölés színével megjeleníti azt a grafikonon.}

\subsubsection*{Megállapítások, megjegyzések}
Ha a \emph{tpmC} ábrán kijelöljük a felső néhány pontot, akkor megállapíthatjuk, hogy a teljes költségek ugyan magasak, de a teljesítmény/ár arány még mindig jónak számít.

\subsection{Milyen adatbázis-kezelő szoftvert válasszunk, ha még mindig az olcsóbb megoldást szeretnénk használni?}
\subsubsection*{Megoldás}
Válasszuk ki a \oszlopnev{Database\_Software} oszlopot, majd \menuelem{Plot $\rightarrow$ Barchart}. A teljes költséget ábrázoló ploton jelöljük ki az alsó pontokat.
\subsubsection*{Megállapítások, megjegyzések}
\Aref{fig:alacsony_TC_adatbazisok}.~ábrán jól látható, hogy melyek azok az adatbázisok, amelyek megfelelnek a választási feltételünknek. Ha egyesével kijelöljük az adatbázisokat a megfelelő grafikonon, akkor megállapítható, hogy az \emph{MS SQL}-es konfigurációk inkább az alsó és középső árkategóriákat fedik le, míg a felső kategória az \emph{Oracle}-re és \emph{IBM}-re jellemző.

\begin{figure}[h!]
  \centering
  \subfloat[Évek - teljes költség, kiemelve az alacsony költségű konfigurációk]{\label{fig:mondrian_Availability_Date_year_Total_Sys_Cost_in_USD_also_pontok}\includegraphics[width=0.30\textwidth]{figures/mondrian_Availability_Date_year_Total_Sys_Cost_in_USD_also_pontok.png}}
  ~~~ % Térköz az ábrák között.
  \subfloat[Az alacsony teljes költségű konfigurációk adatbázisai]{\label{fig:mondrian_Company_barchart_TC_also_kategoria}\includegraphics[width=0.30\textwidth]{figures/mondrian_Company_barchart_TC_also_kategoria.png}}
  \caption{Olcsó konfigurációk adatbázisai}
  \label{fig:alacsony_TC_adatbazisok}
\end{figure}

\subsection{Milyen operációs rendszert válasszunk, ha a teljes költséget minimalizálni akarjuk?}
\subsubsection*{Megoldás}
Az \oszlopnev{Operating\_System} oszlopból készítsünk egy barchartot, majd az \emph{Availability\_Date\_year} - \\ \emph{Total\_Sys\_Cost\_in\_USD} scatterploton jelöljük ki az alsó pontokat. Látható \aref{fig:alacsony_TC_operacios_rendszerek}.~ábrán, hogy főleg az \emph{MS Windows}-os és \emph{Oracle Enterprise Linux}-os konfigurációk a legolcsóbbak a teljes költséget nézve.

\begin{figure}[h!]
  \centering
  \subfloat[Évek - teljes költség, kiemelve az alacsony költségű konfigurációk]{\label{fig:mondrian_Availability_Date_year_Total_Sys_Cost_in_USD_also_pontok}\includegraphics[width=0.30\textwidth]{figures/mondrian_Availability_Date_year_Total_Sys_Cost_in_USD_also_pontok.png}}
  ~~~ % Térköz az ábrák között.
  \subfloat[Az alacsony teljes költségű konfigurációk operációs rendszerei]{\label{fig:mondrian_Operating_System_barchart_also_kategoria}\includegraphics[width=0.30\textwidth]{figures/mondrian_Operating_System_barchart_also_kategoria.png}}
  \caption{Olcsó konfigurációk operációs rendszerei}
  \label{fig:alacsony_TC_operacios_rendszerek}
\end{figure}

\subsubsection*{Megállapítások, megjegyzések}
Az \emph{Availability\_Date\_year} - \emph{tpmC} pontdiagramon kijelölve a felső pontokat látható, hogy a legnagyobb teljesítményt nyújtó gépeken \emph{Unix/Linux} variánsok futnak (\ref{fig:tpmc_felso_kategoria_os}.~ábra), míg az \emph{Operating\_System} barcharton a \emph{Windows}-ra kattintva láthatjuk a \emph{tpmC} ábráján, hogy ezen konfigurációk teljesítménye nem a legjobb.

\begin{figure}[h!]
  \centering
  \subfloat[Évek - teljesítmény, kiemelve a nagy teljesítményű konfigurációk]{\label{fig:mondrian_Availability_Date_year_tpmC_scatterplot_felso_konfigok}\includegraphics[width=0.30\textwidth]{figures/mondrian_Availability_Date_year_tpmC_scatterplot_felso_konfigok.png}}
  ~~~ % Térköz az ábrák között.
  \subfloat[Nagy teljesítményű konfigurációk operációs rendszerei]{\label{fig:mondrian_Company_barchart_felso_tpmC_konfigok}\includegraphics[width=0.30\textwidth]{figures/mondrian_Company_barchart_felso_tpmC_konfigok.png}}
  \caption{Nagy teljesítményű konfigurációk operációs rendszerei}
  \label{fig:tpmc_felso_kategoria_os}
\end{figure}

\subsection{Melyik beszállító melyik adatbázis-kezelőben és operációs rendszerben ,,utazik''?}
\subsubsection*{Megoldás}
A már meglévő \emph{Operating\_System}, \emph{Database\_Software} és \emph{Company} barchartok esetén a beszállítókat végig kattintgatva megvizsgálhatjuk, hogy mely OS-eket és DBMS-eket preferálják. \Aref{fig:company_os_dbms}.~ábráról leolvashatjuk, hogy a HP esetében a Windows operációs rendszereket és az MS SQL adatbázis megvalósításokat preferálják.

\begin{figure}[h!]
  \centering
  \subfloat[Beszállítók, HP kiválasztva]{\label{fig:mondrian_Company_barchart_HP_selected}\includegraphics[width=0.30\textwidth]{figures/mondrian_Company_barchart_HP_selected.png}}
  ~~~ % Térköz az ábrák között.
  \subfloat[A HP által preferált operációs rendszerek]{\label{fig:mondrian_Operating_System_barchart_HP_selected}\includegraphics[width=0.30\textwidth]{figures/mondrian_Operating_System_barchart_HP_selected.png}}
  ~~~ % Térköz az ábrák között.
  \subfloat[A HP által preferált adatbázis-kezelő rendszerek]{\label{fig:mondrian_Database_Software_barchart_HP_selected}\includegraphics[width=0.30\textwidth]{figures/mondrian_Database_Software_barchart_HP_selected.png}}
  \caption{Beszállítók által preferált OS-ek és DBMS-ek}
  \label{fig:company_os_dbms}
\end{figure}

\subsection{Melyik beszállítót, adatbázis-kezelőt és operációs rendszert tartalmazza a leggyakrabban az adathalmaz?}
\subsubsection*{Megoldás}
Vegyünk fel mozaik diagramokat! Ehhez válasszuk ki a \oszlopnev{Company} és \oszlopnev{Database\_Software} oszlopokat, ezután \menuelem{Plot $\rightarrow$ Mosaic Plot}, majd végezzük el ugyanezt a \oszlopnev{Company} és \oszlopnev{Operating\_System} oszlopokra is. A legnagyobb területű téglalapokat ki választva nézhetjük meg, hogy melyik a legelterjedtebb OS, DBMS és beszállító a piacon (\ref{fig:legnagyobb_company_os_dbms}.~ábra).
\subsubsection*{Megállapítások, megjegyzések}
A legnagyobb téglalapot a \emph{HP} - \emph{MS Windows} - \emph{MS SQL} hármas birtokolja, ami valószínűleg nem független az elterjedtségtől sem.

\begin{figure}[h!]
  \centering
  \subfloat[Beszállító - operációs rendszer mozaik diagram]{\label{fig:mondrian_Company_Operating_System_mosaic_plot}\includegraphics[width=0.30\textwidth]{figures/mondrian_Company_Operating_System_mosaic_plot.png}}
  ~~~ % Térköz az ábrák között.
  \subfloat[Beszállító - adatbázis-kezelő rendszer mozaik diagram]{\label{fig:mondrian_Company_Database_Software_mosaic_plot}\includegraphics[width=0.30\textwidth]{figures/mondrian_Company_Database_Software_mosaic_plot.png}}
  \hfill
  \subfloat[Beszállítók]{\label{fig:mondrian_Company_barchart_mosaic_plotbol}\includegraphics[width=0.30\textwidth]{figures/mondrian_Company_barchart_mosaic_plotbol.png}}
  ~~~ % Térköz az ábrák között.
   \subfloat[Operációs rendszerek]{\label{fig:mondrian_Operating_System_barchart_mosaic_plotbol}\includegraphics[width=0.30\textwidth]{figures/mondrian_Operating_System_barchart_mosaic_plotbol.png}}
  ~~~ % Térköz az ábrák között.
   \subfloat[Adatbázis-kezelő rendszerek]{\label{fig:mondrian_Database_Software_barchart_mosaic_plotbol}\includegraphics[width=0.30\textwidth]{figures/mondrian_Database_Software_barchart_mosaic_plotbol.png}}
  \caption{}
  \label{fig:legnagyobb_company_os_dbms}
\end{figure}

\subsection{Próbáljunk összefüggést találni a teljesítmények, költségek, operációs rendszerek és adatbázis-kezelők között!}
\subsubsection*{Megoldás}
Jelöljük ki a \oszlopnev{Company}, \oszlopnev{tpmC}, \oszlopnev{Price\_per\_Perf\_in\_USD}, \oszlopnev{Total\_Sys\_Cost\_in\_USD}, \\ \oszlopnev{Database\_Software},  \oszlopnev{Operating\_System} és \oszlopnev{Currency} oszlopokat (ebben a sorrendben!),\\ majd \menuelem{Plot $\rightarrow$ Parallel Coordinates}. \Aref{fig:mondrian_parallel_coordinates_osszefuggesek}.~ábrán az előző feladatban alkalmazott kijelölés látható (\emph{HP}, \emph{MS Windows}, \emph{MS SQL} és az azokhoz tartozó költség és teljesítmény jellemzők).

\begin{figure}[h!]
\centering
\includegraphics[width=1.00\textwidth]{figures/mondrian_parallel_coordinates_osszefuggesek.png}
\caption{Párhuzamos koordináták az adathalmaz változóiból \label{fig:mondrian_parallel_coordinates_osszefuggesek}}
\end{figure}

\subsubsection*{Megállapítások, megjegyzések}
Az \emph{MSSQL} fajlagosan (tmpC/USD) drága, de az összköltségük ezeknek a konfigurációknak alacsony. A \emph{Windows} alapú rendszerek olcsóbbak, de rosszabb teljesítményt is nyújtanak. Az \emph{AIX} operációs rendszerre csak \emph{DB2}, \emph{Oracle} és hatékonynak mondható adatbázis-kezelők vannak.

A párhuzamos koordináták módszere arra is alkalmas, hogy szelektáljunk a változók között. Pl. a \emph{Currency} oszlopnál látszik, hogy semmihez nincs köze. Fontos a változók sorrendje (pl. \emph{Database\_Software}, \emph{Operating\_System} egymás mellé érdemes helyezni).

\section{Összefoglalás}

\todo{Megírni!}

\end{document}